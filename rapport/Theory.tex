\begin{document}



\section{Theory}

%?kanske forst forklara och gora en fin overgang varfor vi har informationen i den ordningen vi har?
%jo kan kanske vara bra
In this section we wil discuss the theory behind equality saturation and the acompaning concepts. We
start by defining an equivalence for our language that will be used by our axioms.
This allows us to save space since instead of storing the subterms to a term, we
store a reference to the equivalence class of those subterms. This allows us to represent a wide 
variety of different combinations of terms efficiently.

We will also define a relation
that will denote the dependencies of the different classes. This will be useful for us to know where to
or rather where not to look for terms to apply axioms to.
\subsection{Equivalence classes}

%In this paper we will define a equivalence class to 
%In this paper we will use the standard equivalence class representation, that is:

For our purpose we will say that two terms are equal if they evaluate to the same value, i.e the language
is referential transparent. So if two terms $t_1$ and $t_2$ are equal it is safe to exchange an occurence
of $t_1$ with $t_2$ and vice versa. This leads to this definition of the relation:
$t_1 \sim t_2 \equiv eval(t_1) = eval(t_2)$


Even if we don't know how the eval function is defined we can deduce that $\sim$ is an equivalence relation.
This because $=$ is an equivalence relation, for example to proof of reflexivity is:
$eval(t) = eval(t) \implies t \sim t$.


Usually we will talk about classes instead of individual terms, these classes will
be named \#1, \#2 and so forth. This corresponds to the usual definition by
$y \in \#1 = [y] = \{ x : x \sim y\}$. From this we get the insight that we should
not merly note that two terms are equal, but we should merge the two different classes.
For example if $x + y \in \#1$ and $y + x \in \#2$ then the axiom of commutitativity
states that the classes \#1 and \#2 really are the same and should be merged.





%There are different factors to consider here, p1 or p2 could both generate side effects that eval can't measure or one of them could take forever to compute (1/2 + 1/4 + .... == 1).

%say something about pure...




\subsection{A term language}

%To futher our understanding of representating a program as a class we should take a look at a language.
%To understand how to 
%The next step to representate programs as equivalence classes is to discuss a example language. 

Let us give an example of how terms may be represented with classes, suppose the
language was the following:

\begin{equation*}
\begin{aligned}
\text{Term} \quad & t_1,t_2 \quad & ::= \; \val & \\
 & & \; | \; t_1 \op t_2  & \quad \text{Compound expression} \\
\end{aligned}
\end{equation*}

This is a very basic language, with one binary operator $\op$ and one 'value` $\val$. 
A sample term could look like
$t_1 = \val \op (\val \op \val)$
which isn't necessarily the same as 
$t_2 = (\val \op \val) \op \val$
without knowing more about the evaluation function.
So we need to assume that they are different and creates two different equivalence
classes \#1 and \#2 with ($t_1 \in \#1$, $t_2 \in \#2$).

This is however not optimal since the subterm ($\val \op \val$) exists
in both the term of the different classes. We don't want such waste of space and 
further more if we were to learn some more information about that term it
would only apply to one of the classes, even though both could potentially benefit from it. 

%it is often hard to see if two seemly lagre programs are same from the top, we have to try and divide the programs in smaller parts without losing any information.

Instead every subexpression in term is to be replaced with a class of terms that are all equivalent.
So for the example above we get these classes:

\begin{equation*}
\begin{aligned}
\#1 & = \{\val\} \\
\#2 &= \{\#1 \op \#1\} \\
\#3 &= \{ \#1 \op \#2\} \\
\#4 &= \{ \#2 \op \#1\}
\end{aligned}
\end{equation*}

where $t_1 \in \#3$ and $t_2 \in \#4$. We now share the common terms and if we 
where to update a class, all classes that depend on that class may use the additional information. %it all classes that depend on it can use the new information.
%This also allows us to in a efficient manner store multiple versions of the same
% terms.
This also allows us to store multiple versions of the same term in a efficient 
manner.

\subsection{The relation $\lhd$ of dependencies}
To save having to go through every class and looking for terms to apply axioms to
we add a notion that the class can change. So a class changes if it gets merged
with another class or a term is added to it. That means we should only look in a
class if it has changed or one of the classes that the class depend on has changed.
This is a relation $x \lhd y$ which means that $y$ depend
on $x$, which is to be defined as:  

\begin{equation*}
x \lhd y \equiv \exists (t_1 \op t_2) \in y \text{ such that } t_1 \in x
\vee t_2 \in x
\end{equation*}

Note that if we have the following class $[0] = \{0, 0 + 0\}$ gives us $[0] \lhd [0]$
which means that a class may depend on it self. We can also have cycles in the depenency
for example:
\begin{equation*}
\begin{aligned}
\quad [2] = \{ 2 , [5] - [3] \} \\ % \quad needed to display [2]
[3] = \{ 3 , [5] - [2] \} \\
[5] = \{ 5 , [2] + [3] \}
\end{aligned}
\end{equation*}

Here all the classes depend on each other which creates a cycle, which is something
to be cautios of. We will generalise this relation
further and have a family of relations indexed by the depth of the dependency.
We will denote this relation by $x \lhd^n y$ which have the following inductive
definition where n is the depth.
\begin{equation*}
\begin{aligned}
x \lhd^1 y &\equiv x \lhd y \\
x \lhd^n y &\equiv \exists (t_1 \op t_2) \in y \text { such that } 
x \lhd^{n-1} t_1 \vee x \lhd^{n-1} t_2 \\
& \text{ or } x \lhd^{n-1} y
\end{aligned}
\end{equation*}

\subsection{Equality Saturation}
To bring it all together for a optimisation framework we will discuss how the optimisation
will work. The first thing we do is taking the language and create a similar language
but where the previous language where recursive we instead replace these with a
representation of our classes. So in the language theory we have above, $\op$ will
not contain terms but represents for the class of its children.

When we have the term rewritten into this new language which is like an equality
graph. We start to look through all rules and classes and try to find new terms,
this will be done by our pattern matching alogrithm described in the next section.

We will continue to do this until it is saturated, i.e. there exist no more rules
to apply, or if we run into a predefined limit of applications. We need to have a
limit because not all terms can be saturated (under a given set of rules), it may
exist rules that will always be applicative for example $\forall \,x. \; x \leadsto x * 1$
. If we would have a rule like this it would always create new terms.

The final step is to go through all classes and pick the best term from this class
to do this we will use a value function to decide which terms are better. Caution
need to be taken since the classes can have a depenency that contain cycles. The
final term is then simply the best from its class.

%* equivalence classes
%* abstrakt model of Expr
%på tal om det, har funderat på om det inte varit smartare (vi kommer inte ändra)
%att ha data TExpr t = Lit Literal | Var String | BinOp BinOp t t | TriOp TriOp t t t
%det hade kunnat göra Rules enklare. Slut parentes. Japp, var abstraka model kan ju se lite ut som det :p
%:)
%* <|
%  * can contain loops
%* monotone fitness function $\forall x,y. x < y ==> f (x + c) < f (y + c)$
%* why saturation is hard (it gets big fast)

\end{document}

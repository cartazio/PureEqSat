\begin{document}
\section{Background}
When writing an optimiser for a compiler there are usually a myriad of different optimisations to consider.
%When writing an optimiser for a compiler usually we have a lot of different 
%optimisations at mind.
For example in a compiler backend you may find peep-hole\cite{peephole}
optimisations, loop-induction variables\cite{loop-induction} and a plethora of others. In most cases 
they are orthogonal in how they optimise, and can benefit from each others optimisations.

While this is common knowledge the reverse also holds true, applying an optimisation 
%This is standard wisdom, but the reverse is also true, applying an optimisation
may destroy an oppurtunity to apply yet another one. If we take the example of having a 
term, $ 2 * x $, a peephole optimiser could rewrite this to the better term 
$ x << 1$. But suppose this term was used in the loop, that the induction variable
depends on. Then it may not be able to do loop induction optimisation and we end up
with a result that is suboptimal.%worse than if we would not.
\paragraph{}
To circumvent this we must carefully choose in which order we apply the optimisations
but even with greatest care we can't win. Different programs benefit differently
on different orders. This make us search for another approach to apply optimisations
that doesn't depend on any order. 
\paragraph{}
We failed with the peep-hole case due to the old version of code being gone when 
the optimisation had been applied.
%The reason for why we failed with the peep-hole case is because when it has been applied,the old version of the code is gone.
And because of this destructive update, each
optimisation has to perform local checks to see if the optimisation would be beneficial in the long run %to apply or not.
These checks are non-trivial to write, so wouldn't it better to not do destruvtive
updates at all?

We follow the approach of Equality Saturation\cite{eqSat}, but we do some things
differently from their paper. The approach is three-fold:

\begin{enumerate}
  \item Translate the term to our equality-representation.
  \item Try to apply as many optimisations as possible, but only add new terms and
        record the equivalence between terms.
  \item Go through the equivalence-graph and pick out the best term.
\end{enumerate}

The language we work with is a small arithmetical language and we have referential
transparency which makes it easier for us to reason about the terms. All the optimisations
we will discuss can be thought of as rewrite rules, and this is why we will hencefor
call them axioms.


\end{document}
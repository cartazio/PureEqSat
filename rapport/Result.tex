\begin{document}
\newcommand{\defBNF}[4] {\text{#1}\quad&#2&::=&\;#3&\text{#4}}
\newcommand{\defaltBNF}[2] {&&|&\;#1&\text{#2}}
\newcommand{\sepBNF}{\;|\;} 
\section{Results}

We chose a simple arithmetical language to experiment with, the reason for this
is that it doesn't contain any constructs that binds variables and is a natural
fit for reasoning by axioms.


\begin{equation*}
\begin{aligned}
\defBNF{Literal}{l}{n \sepBNF b}{} \\
\defBNF{Term}{t}{l \sepBNF t + t \sepBNF t * t \sepBNF t == t \sepBNF t \&\& t
                 \sepBNF t \| t \sepBNF \text{if } t \text{ then } t \text{ else } t
                 }{} 
\end{aligned}
\end{equation*}

This simple language has the obvious semantic with very natural eval function
that simply maps each term to the corresponding value that we expect. The axioms
follows from the mathematical reasoning basically this is how we use the rules
in our implementation.

\begin{verbatim}
rules :: [Rule]
rules = [ assoc    rAdd
        , assoc    rMul
        , assoc    rAnd
        , assoc    rOr
        , commute  rAdd
        , commute  rMul
        , commute  rAnd
        , commute  rOr
        , commute  rEq
        , identity rAdd (rInt 0)
        , identity rMul (rInt 1)
        , identity rAnd rTrue
        , identity rOr  rFalse
        , zero     rMul (rInt 0)
        , zero     rAnd rFalse
        , zero     rOr  rTrue

        , forall3 $ \x y z -> rMul x (rAdd y z) ~> rAdd (rMul x y) (rMul x z)
        , forall3 $ \x y z -> rAdd (rMul x y) (rMul x z) ~> rMul x (rAdd y z)
        , forall1 $ \x -> x `rEq` x ~> rTrue
        , forall3 $ \x y z -> ((x `rAdd` z) `rEq` (y `rAdd` z)) ~> (x `rEq` y)
        , forall2 $ \x y -> rIf (rTrue) x y ~> x
        , forall2 $ \x y -> rIf (rFalse) x y ~> y
        , forall2 $ \x y -> rIf y x x ~> x
        ]

zero op v     = forall1 $ \x -> (x `op` v) ~> v
identity op v = forall1 $ \x -> (x `op` v) ~> x
commute op    = forall2 $ \x y -> (x `op` y) ~> (y `op` x)
assoc op      = forall3 $ \x y z -> ((x `op` y) `op` z) ~> (x `op` (y `op` z))
\end{verbatim}

Remember that when we do unification on patterns, we do not check that the terms
are equal but we check if they relate. So for example the rule $\forall \, x. x == x \leadsto \text{True}$
should really be seen as $\forall \, x \, y. x \sim y \implies x == y \leadsto \text{True}$.
So if we can prove that $x$ and $y$ will evaluate to the same value, then the equality
will evaluate to True. For example, $x + y == y + x$ will be the same as True, even
if we don't know what $x$ and $y$ are.

It is also worth mentioning that we also have rules for evaluating terms if all
the subterms are known. So if we see an expression like $7 * 191$ we will say that
it is equal to the term $1337$. Some of these evaluation rules have been expressed
in the axioms that was given above.






* cool examples:  x + y == y + x, show number of rules
* Some timing of examples.. <- what sort of time should we take?
-- we can't compare the results against the pre-optimized, we only have our cost function.
a sure, cost function result.
* Simple ordo ananlysis? for parts of algorithm.
* issues with scope? <-- resultat eller vart?
\end{document}
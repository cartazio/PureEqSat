\begin{document}
\section{Results}

We chose a simple arithmetical language to experiment with, the reason for this
is that it doesn't contain any constructs that binds variables and is a natural
fit for reasoning by axioms.


\begin{equation*}
\begin{aligned}
\defBNF{Literal}{l}{n \sepBNF b}{} \\
\defBNF{Term}{t}{l \sepBNF t + t \sepBNF t * t \sepBNF t == t \sepBNF t \&\& t
                 \sepBNF t \| t \sepBNF \text{if } t \text{ then } t \text{ else } t
                 }{} 
\end{aligned}
\end{equation*}

This language has a very natural semantic in which the eval function simply maps
each term to the corresponding mathematical value. 
The rules we use follow from the mathematical axioms used for these objects, and
bellow we will list some of them.

\begin{verbatim}
rules :: [Rule]
rules = [ assoc    rAdd
        , assoc    rMul
        , assoc    rAnd
        , assoc    rOr
        , commute  rAdd
        , commute  rMul
        , commute  rAnd
        , commute  rOr
        , commute  rEq
        , identity rAdd (rInt 0)
        , identity rMul (rInt 1)
        , identity rAnd rTrue
        , identity rOr  rFalse
        , zero     rMul (rInt 0)
        , zero     rAnd rFalse
        , zero     rOr  rTrue

        , forall3 $ \x y z -> rMul x (rAdd y z) ~> rAdd (rMul x y) (rMul x z)
        , forall3 $ \x y z -> rAdd (rMul x y) (rMul x z) ~> rMul x (rAdd y z)
        , forall1 $ \x -> x `rEq` x ~> rTrue
        , forall3 $ \x y z -> ((x `rAdd` z) `rEq` (y `rAdd` z)) ~> (x `rEq` y)
        , forall2 $ \x y -> rIf (rTrue) x y ~> x
        , forall2 $ \x y -> rIf (rFalse) x y ~> y
        , forall2 $ \x y -> rIf y x x ~> x
        ]

zero op v     = forall1 $ \x -> (x `op` v) ~> v
identity op v = forall1 $ \x -> (x `op` v) ~> x
commute op    = forall2 $ \x y -> (x `op` y) ~> (y `op` x)
assoc op      = forall3 $ \x y z -> ((x `op` y) `op` z) ~> (x `op` (y `op` z))
\end{verbatim}

Remember that when we do unification on patterns, we do not check that the terms
are equal but we check if they relate. So for example the rule $\forall \, x. x == x \leadsto \text{True}$
should really be seen as $\forall \, x \, y. x \sim y \implies x == y \leadsto \text{True}$.
So if we can prove that $x$ and $y$ will evaluate to the same value, then the equality
will evaluate to True. For example, $x + y == y + x$ will be the same as True, even
if we don't know what $x$ and $y$ are.

It is also worth mentioning that we also have rules for evaluating terms if all
the subterms are known. So if we see an expression like $7 * 191$ we will say that
it is equal to the term $1337$. Some of these evaluation rules have been expressed
in the axioms that was given above.

\paragraph{Examples}

Let us look at some examples of how these rules interact which each-other. First
we have a small example in which only the commutativity and the cancel law works
together to simplify terms.

\begin{verbatim}
main = z + y == x + z
\end{verbatim}

This will be optimized to term \verb|main = y == x| which is what you expect, this
is represented by six classes. Three for the different unknown variables, two
for the different additions and the last one is for the equality. 

Our next example will be:

\begin{verbatim}
main = if x == a then x * (a + b) else b * x + x * a
\end{verbatim}

Here we don't know what the equality is between x and a but we notice that the branches
are equal so the final term is \verb"main = x * (a + b)" this is something which
not even gcc can find on any optimization level as will be seen below.

\paragraph{Comparing with gcc/java}
One interesting question to pose is how well our optimizer fairs against the old giants.
To make a fair comparison we need to build an expression that is equivalent to an expression in our language. This is easily done be creating a function taking arguments\footnote{We use ints since they are primitive in both Java and c} and then see how that function is optimized. For gcc we just inspect the assembly code and for java we decompile the byte code.
%\paragraph{}%Round 1}
\begin{itemize}
  \item \verb|if True then 1 else 2|
\end{itemize}
Both gcc\footnote{compiled with gcc -On, n=\{1,2,3\}} and java \footnote{openjdk6} optimizes this to $1$.

\begin{itemize}
  \item \verb|if x + y == y + x then 1 else 2|
\end{itemize}
Here the java version fails to optimize and create two separate branches. While the c version still optimizes the code to $1$.
\begin{itemize}
  \item \verb|if v then x * a + x * b else x * (a + b)|
\end{itemize}
This is the same as our second example above, but the branches has been exchanged with each other.
Gcc fails to realize that the two branches are in fact equal and compiles the following code: 
\begin{verbatim}
int fun(int x, int a, int b, int v)                                                         {                                                                                    
   if(v)
   {                                                                     
      return x * a + x * b;                                                        
   } else {                                                                         
      return x * (a + b);                                                          
   }
}
\end{verbatim}

to roughly the following assembly. 
\begin{verbatim}
fun:                                                                                 
  testl   %ecx, %ecx                                                               
  jne .L5                                                                          
  leal    (%rdx,%rsi), %eax                                                        
  imull   %edi, %eax                                                               
  ret                                                                              
  .L5:                                                                                 
    leal    (%rdx,%rsi), %eax                                                        
    imull   %edi, %eax                                                               
    ret                                                                              
\end{verbatim}
Note the jne instruction, this was
compiled with gcc -O3 -s, and the asm still contains a compare.
This while our optimizer solves each of the above cases.

\subsection{Problem with pattern matching}

Sadly we have noticed some problems with our pattern matcher, and the way in which
we apply the rules. Currently we take a class if it has changed we try to apply
all rules (of depth that matches the depth of change). This unfortunately makes it
so that in some cases some rules starts creating a lot of new classes, most of which 
already exists but it is some later rule that will realize this equality.

This makes for an explosion of new classes with only one term (and a term in the
original class) and all of these new classes are used by other rules to create even
more classes. And much of these classes we already have represented we just don't
find that out before it is to late and we have to many terms.

We\footnote{The authors of this paper} did not have time to really explore 
this issue in detail, and have therefore have
no real solution to this problem. The reason for not finding this out earlier is
that we previously had a bug in the pattern matcher that combined the different branches
wrong and pruned away a lot of them. So we had a fast solution but we could miss
to apply rules. We found out about this just a few days before the deadline, c'est
la vie.

%* cool examples:  x + y == y + x, show number of rules
%* Some timing of examples.. <- what sort of time should we take?
%-- we can't compare the results against the pre-optimized, we only have our cost function.
%a sure, cost function result.
%* Simple ordo ananlysis? for parts of algorithm.
%* issues with scope? <-- resultat eller vart?
\end{document}
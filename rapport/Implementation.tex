\begin{document}
\section{Implementation}
In this section we will take a closer look at the different aspects that relate
to the implementation of the equality saturation. We will start of with how
we store the classes, and then we will explain how we find the equivalences by 
the use of axioms. Finaly we talk about how we choose the best representation in
a given class.
\subsection{How do we keep track of the classes?}
To keep track of the the classes we must have an efficient data-structure to store
them in. The different operations we may perform on classes is of the followin:

\begin{description}
  \item[makeClass] Given a term, we create a class where it belongs to.
  \item[union]     Given two classes we may find that they are in fact equal, then
                   we need to merge them.
  \item[addElem]   Sometimes we find new terms that we know in which class they
                   should belong to.
\end{description}

Due to these requirements we have opted for a union-find structure and acompaning
algorithms. We have added some structure to it, for example we store all the terms
in the root node.

* union-find
* * limitation in ordinary union-find (can't get elements etc)
\subsection{Pattern-Matching on terms to see find axioms to apply}
* * looking throw every term for match in the class. (hint callback future work) 
* dirty-bit work
  * depth of change in <|
\subsection{Fitness function}
* cost-method (kanske theory), dynamic programing delen är implementation.
\end{document}